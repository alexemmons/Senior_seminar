% This is a sample document using the University of Minnesota, Morris, Computer Science
% Senior Seminar modification of the ACM sig-alternate style. Much of this content is taken
% directly from the ACM sample document illustrating the use of the sig-alternate class. Certain
% parts that we never use have been removed to simplify the example, and a few additional
% components have been added.

% See https://github.com/UMM-CSci/Senior_seminar_templates for more info and to make
% suggestions and corrections.

\documentclass{sig-alternate}
\usepackage{color}
\usepackage[colorinlistoftodos]{todonotes}

%%%%% Uncomment the following line and comment out the previous one
%%%%% to remove all comments
%%%%% NOTE: comments still occupy a line even if invisible;
%%%%% Don't write them as a separate paragraph
%\newcommand{\mycomment}[1]{}

\begin{document}

% --- Author Metadata here ---
%%% REMEMBER TO CHANGE THE SEMESTER AND YEAR
\conferenceinfo{UMM CSci Senior Seminar Conference, December 2014}{Morris, MN}

\title{Automatic Chord Recognition from Audio}

\numberofauthors{1}

\author{
% The command \alignauthor (no curly braces needed) should
% precede each author name, affiliation/snail-mail address and
% e-mail address. Additionally, tag each line of
% affiliation/address with \affaddr, and tag the
% e-mail address with \email.
\alignauthor
Alex R. Emmons\\
	\affaddr{Division of Science and Mathematics}\\
	\affaddr{University of Minnesota, Morris}\\
	\affaddr{Morris, Minnesota, USA 56267}\\
	\email{emmon046@morris.umn.edu}
}

\maketitle
\begin{abstract}
This paper will look at the process of chord recognition from audio. Areas discussed will include different methods of signal processing and feature extraction, hidden Markov models, and the effects of pre/post processing. This research is used in the area of Music Information Retrieval to document and categorize music.
\end{abstract}

\keywords{Automatic Chord Recognition, HMM, Hidden Markov Models}

\section{Introduction}
\todo[inline]{TODO}

\section{Hidden Markov Models}\label{main}

A hidden Markov model (HMM) is a statistical model which describes a finite set of states, in this case chords, each with a probability distribution. Transitions between these states are governed by a set of transition probabilities that describe the likelihood of transitioning from one to another. \cite{TaeMin:2014} HMMs are used in a wide variety of pattern recognition environments such as speech, handwriting, and gesture recognition, as well as in bioinformatics.  

\subsection{HMM with Audio-From-Symbolic Data}

In the HMM trained with audio-from-symbolic data \cite{Lee:2006} a 36-state HMM is used, with each state representing a single chord. Using an ergodic model, which allows every possible transition from chord to chord, the model parameters are learned and then the Viterbi algorithm is applied. The Viterbi algorithm finds the most likely path, or chord sequence, by restricting unlikely chord transitions. \cite{TaeMin:2014} Two things are needed to train this model: chord label files, and audio data. In this case they are both being generated from the same symbolic data. The first step is to use a chord analysis tool to generate a file with complete chord information for a piece of music. Using the same symbolic data, the audio files are generated using a sample-based synthesizer. This audio data is in perfect sync with the chord label file and is just as good as a real recording because it contains the upper harmonics that would be generated from real instruments. \cite{Lee:2006} 

%\subsection{Math Equations}
%You may want to display math equations in three distinct styles:
%inline, numbered or non-numbered display.  Each of
%the three are discussed in the next sections.

%\subsubsection{Inline (In-text) Equations}
%A formula that appears in the running text is called an
%inline or in-text formula.  It is produced by the
%\textbf{math} environment, which can be
%invoked with the usual \texttt{{\char'134}begin. . .{\char'134}end}
%construction or with the short form \texttt{\$. . .\$}. You
%can use any of the symbols and structures,
%from $\alpha$ to $\omega$, available in
%\LaTeX\cite{Lamport:LaTeX}; this section will simply show a
%few examples of in-text equations in context. Notice how
%this equation: \begin{math}\lim_{n\rightarrow \infty}x=0\end{math},
%set here in in-line math style, looks slightly different when
%set in display style.  (See next section).
%
%\subsubsection{Display Equations}
%A numbered display equation -- one set off by vertical space
%from the text and centered horizontally -- is produced
%by the \textbf{equation} environment. An unnumbered display
%equation is produced by the \textbf{displaymath} environment.
%
%Again, in either environment, you can use any of the symbols
%and structures available in \LaTeX; this section will just
%give a couple of examples of display equations in context.
%First, consider the equation, shown as an inline equation above:
%%%%%
%\begin{equation*}
%\lim_{n\rightarrow \infty}x=0
%\end{equation*}
%%%%%
%Notice how it is formatted somewhat differently in
%the \textbf{displaymath}
%environment.  Now, we'll enter an unnumbered equation:
%\begin{displaymath}\sum_{i=0}^{\infty} x + 1\end{displaymath}
%and follow it with another numbered equation:
%\begin{equation}\sum_{i=0}^{\infty}x_i=\int_{0}^{\pi+2} f\end{equation}
%just to demonstrate \LaTeX's able handling of numbering.
%
%\subsection{Multi-line formulas}
%%%%%%% \[ \] denotes math; array environment is needed for multi-line
%%%%%%% format, {c} means that the array has one column, centered 
%%%%%%  (alternatives are {l} for left-aligned, {r} for right-aligned)
%%%%%%  \\ denotes a new line
%\[
%\begin{array}{c}
%n_1 = \sum_{i = 1}^k a_i \\
%n_2 = \prod_{i = 1}^k b_i
%\end{array}
%\]
%
%\subsection{Citations}
%Citations to articles \cite{Morman:2006,Lee:2006,TaeMin:2014} listed
%in the Bibliography section of your
%article will occur throughout the text of your article.
%You should use BibTeX to automatically produce this bibliography;
%you simply need to insert one of several citation commands with
%a key of the item cited in the proper location in
%the \texttt{.tex} file \cite{OM:2008}.
%The key is a short reference you invent to uniquely
%identify each work; in this sample document, the key is
%the first author's surname and a
%word from the title.  This identifying key is included
%with each item in the \texttt{.bib} file for your article.
%
%The details of the construction of the \texttt{.bib} file
%are beyond the scope of this sample document, but more
%information can be found in the \textit{Author's Guide},
%and exhaustive details in the \textit{\LaTeX\ User's
%Guide}.
%
%This article shows only the plainest form
%of the citation command, using \texttt{{\char'134}cite}.
%This is what is stipulated in the SIGS style specifications.
%No other citation format is endorsed or supported.
%
%\subsection{Tables}
%Because tables cannot be split across pages, the best
%placement for them is typically the top of the page
%nearest their initial cite.  To
%ensure this proper ``floating'' placement of tables, use the
%environment \textbf{table} to enclose the table's contents and
%the table caption.  The contents of the table itself must go
%in the \textbf{tabular} environment, to
%be aligned properly in rows and columns, with the desired
%horizontal and vertical rules.  Again, detailed instructions
%on \textbf{tabular} material
%is found in the \textit{\LaTeX\ User's Guide}.
%
%Immediately following this sentence is the point at which
%Table 1 is included in the input file; compare the
%placement of the table here with the table in the printed
%dvi output of this document.
%
%\begin{table}
%\centering
%\caption{Frequency of Special Characters}
%\begin{tabular}{|c|c|l|} \hline
%Non-English or Math&Frequency&Comments\\ \hline
%\O & 1 in 1,000& For Swedish names\\ \hline
%$\pi$ & 1 in 5& Common in math\\ \hline
%\$ & 4 in 5 & Used in business\\ \hline
%$\Psi^2_1$ & 1 in 40,000& Unexplained usage\\
%\hline\end{tabular}
%\end{table}
%
%To set a wider table, which takes up the whole width of
%the page's live area, use the environment
%\textbf{table*} to enclose the table's contents and
%the table caption.  As with a single-column table, this wide
%table will ``float" to a location deemed more desirable.
%Immediately following this sentence is the point at which
%Table 2 is included in the input file; again, it is
%instructive to compare the placement of the
%table here with the table in the printed dvi
%output of this document.
%
%
%\begin{table*}
%\centering
%\caption{Some Typical Commands}
%\begin{tabular}{|c|c|l|} \hline
%Command&A Number&Comments\\ \hline
%\texttt{{\char'134}alignauthor} & 100& Author alignment\\ \hline
%\texttt{{\char'134}numberofauthors}& 200& Author enumeration\\ \hline
%\texttt{{\char'134}table}& 300 & For tables\\ \hline
%\texttt{{\char'134}table*}& 400& For wider tables\\ \hline\end{tabular}
%\end{table*}
%% end the environment with {table*}, NOTE not {table}!
%
%\subsection{Figures}
%Like tables, figures cannot be split across pages; the
%best placement for them
%is typically the top or the bottom of the page nearest
%their initial cite.  To ensure this proper ``floating'' placement
%of figures, use the environment
%\textbf{figure} to enclose the figure and its caption.
%
%This sample document contains examples of %\textbf{.eps}
%%and 
%a \textbf{.pdf} file to be displayable with \LaTeX.  More
%details on each of these is found in the \textit{Author's Guide}.
%
%\begin{figure}
%\centering
%\psfig{file=sample_graph.pdf,width =3in}
%\caption{A sample graph just spanning one column.}
%\end{figure}
%
%
%As was the case with tables, you may want a figure
%that spans two columns.  To do this, and still to
%ensure proper ``floating'' placement of tables, use the environment
%\textbf{figure*} to enclose the figure and its caption.
%\begin{figure*}
%\centering
%\psfig{file=sample_graph.pdf,width =5.3in}
%\caption{A sample graph that needs to span two columns of text.}
%\end{figure*}
%and don't forget to end the environment with
%{figure*}, not {figure}!
%
%It's easiest and you tend to get the best quality if your figures vector graphics
%in PDF format. You can include other formats such as PNG, but they will usually
%not look nearly as professional, especially when printed on high resolution printers.
%\emph{Be vary wary of screen captures from other papers. They tend to look pixelated
%and amateurish even at high resolutions.}
%
%\subsection{Theorem-like Constructs}
%Other common constructs that may occur in your article are
%the forms for logical constructs like theorems, axioms,
%corollaries and proofs.  There are
%two forms, one produced by the
%command \texttt{{\char'134}newtheorem} and the
%other by the command \texttt{{\char'134}newdef}; perhaps
%the clearest and easiest way to distinguish them is
%to compare the two in the output of this sample document:
%
%This uses the \textbf{theorem} environment, created by
%the\linebreak\texttt{{\char'134}newtheorem} command:
%\newtheorem{theorem}{Theorem}
%\begin{theorem}
%Let $f$ be continuous on $[a,b]$.  If $G$ is
%an antiderivative for $f$ on $[a,b]$, then
%\begin{displaymath}\int^b_af(t)dt = G(b) - G(a).\end{displaymath}
%\end{theorem}
%
%The other uses the \textbf{definition} environment, created
%by the \texttt{{\char'134}newdef} command:
%\newdef{definition}{Definition}
%\begin{definition}
%If $z$ is irrational, then by $e^z$ we mean the
%unique number which has
%logarithm $z$: \begin{displaymath}{\log e^z = z}\end{displaymath}
%\end{definition}
%
%Two lists of constructs that use one of these
%forms is given in the
%\textit{Author's  Guidelines}.
% 
%There is one other similar construct environment, which is
%already set up
%for you; i.e. you must \textit{not} use
%a \texttt{{\char'134}newdef} command to
%create it: the \textbf{proof} environment.  Here
%is a example of its use:
%\begin{proof}
%Suppose on the contrary there exists a real number $L$ such that
%\begin{displaymath}
%\lim_{x\rightarrow\infty} \frac{f(x)}{g(x)} = L.
%\end{displaymath}
%Then
%\begin{displaymath}
%l=\lim_{x\rightarrow c} f(x)
%= \lim_{x\rightarrow c}
%\left[ g{x} \cdot \frac{f(x)}{g(x)} \right ]
%= \lim_{x\rightarrow c} g(x) \cdot \lim_{x\rightarrow c}
%\frac{f(x)}{g(x)} = 0\cdot L = 0,
%\end{displaymath}
%which contradicts our assumption that $l\neq 0$.
%\end{proof}
%
%Complete rules about using these environments and using the
%two different creation commands are in the
%\textit{Author's Guide}; please consult it for more
%detailed instructions.  If you need to use another construct,
%not listed therein, which you want to have the same
%formatting as the Theorem
%or the Definition\cite{salas:calculus} shown above,
%use the \texttt{{\char'134}newtheorem} or the
%\texttt{{\char'134}newdef} command,
%respectively, to create it.
%
%\subsection*{A {\secit Caveat} for the \TeX\ Expert}
%Because you have just been given permission to
%use the \texttt{{\char'134}newdef} command to create a
%new form, you might think you can
%use \TeX's \texttt{{\char'134}def} to create a
%new command: \textit{Please refrain from doing this!}
%Remember that your \LaTeX\ source code is primarily intended
%to create camera-ready copy, but may be converted
%to other forms -- e.g. HTML. If you inadvertently omit
%some or all of the \texttt{{\char'134}def}s recompilation will
%be, to say the least, problematic.
%
%\section{Conclusions}
%This paragraph will end the body of this sample document.
%Remember that you might still have Acknowledgments or
%Appendices; brief samples of these
%follow.  There is still the Bibliography to deal with; and
%we will make a disclaimer about that here: with the exception
%of the reference to the \LaTeX\ book, the citations in
%this paper are to articles which have nothing to
%do with the present subject and are used as
%examples only.
%
%\section{Acknowledgments}
%
%This section is optional; it is a location for you
%to acknowledge grants, funding, editing assistance and
%what have you.
%
%It is common (but by no means necessary) for students to thank
%their advisor, and possibly other faculty, friends, and family who provided
%useful feedback on the paper as it was being written.
%
%In the present case, for example, the
%authors would like to thank Gerald Murray of ACM for
%his help in codifying this \textit{Author's Guide}
%and the \textbf{.cls} and \textbf{.tex} files that it describes.

% The following two commands are all you need in the
% initial runs of your .tex file to
% produce the bibliography for the citations in your paper.
\bibliographystyle{abbrv}
% sample_paper.bib is the name of the BibTex file containing the
% bibliography entries. Note that you *don't* include the .bib ending here.
\bibliography{sample_paper}  
% You must have a proper ".bib" file
%  and remember to run:
% latex bibtex latex latex
% to resolve all references

\end{document}
